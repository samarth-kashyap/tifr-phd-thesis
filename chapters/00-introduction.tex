% !TEX root = ../thesis.tex
\chapter{Introduction}
\label{chap:00-introduction}

The Sun is a main-sequence star that can be spatially resolved
and investigated in great detail, thanks to its
proximity to us. A greater understanding of stellar structure
and interior can be sought, by using the information of 
solar structure and interior as a benchmark. Nuclear reactions
at the solar interior produce energy, which are transmitted 
out through the optically thin outer surface of the Sun 
through electromagnetic radiation. This radiation enables 
the observation of the solar surface and it reveals the rich
dynamics of the Sun's evolution -- from time-scales of seconds
to decades. The subsurface of the Sun being optically thick, 
obfuscates an easy understanding of the solar interior. However, 
solar oscillations, which propagate through the solar interior
and show up on the solar surface, are sensitive to the conditions 
of the solar interior. A detailed analysis of these oscillations
has enabled a suprisingly precise imaging of the solar interior. 

Stellar models are typically sherically symmetric i.e. the 
properties are purely a function of the stellar radius.
The ingredients that go into modelling a star, which 
affect the structure and evolution are 
its mass $m$, initial helium abundance $Y_0$ and initial
heavy metal abundace $Z_0$. Convection in stars is modelled 
by employing the mixing-length parameter $\alpha$. Such a 
model is evolved over time to obtain to reach the observed
temperature and luminosity. For the Sun, independent 
estimates of mass, radius, luminosity and age exist, which 
can be used to determine the helium abundance $Y_0$ and 
the mixing-length parameter $\alpha$, iteratively. Such a 
solar model is termed as a standard solar model (SSM) and
a popular choice of an SSM is Model S \citep{JCD-1996-Science}.
Such a model is spherically symmetric, non-rotation, non-magnetic,
adiabatic, isotropic and static (SNRNMAIS). 
However, we have observed the Sun to be magnetically active, rotating and highly 
dynamic. It is also to be noted that the SSMs reproduce the 
observed solar oscillation power spectrum quite well --
thus enabling a perturbative analysis of deviations from Model S. 
This means that propagation of seismic waves in the solar interior
can be used to measure flows in the interior, given that the 
flows are not strong enough to alter the physics of wave propagation
itself. In this thesis, we explore and establish the method of 
estimating flows in the solar interior through the measurement of
distortion of helioseismic modes.

Solar activity is the term used to describe a range of magnetic phenomena, both short and
long-lived, such as flares, sunspots, coronal mass ejections, etc
\citep{Usoskin-2017-LRSP}. Solar activity is known to periodically
alternate between periods of high and low acitivity --
well characterized by various indices of acitivity, such as
the \emph{global sunspot number} \citep[GSN;][]{Hoyt-1998-SoPh},
the \emph{flare index} \citep{Ozguc-2003-SoPh},
the F10.7 \emph{index} \citep{Tapping-1994-SoPh},
\emph{sunspot area} \citep{Baranyi-2001-MNRAS}, etc. The period
of this activity cycle (\emph{solar cycle}) is approximately 11 years
\citep{Hathaway-2015-LRSP}. The strength and variation of observed solar
activity is thought to be driven by flow fields in the convective
envelope -- particularly differential rotation, meridional circulation
and convection \citep{Charbonneau-2020-LRSP, Fan-2009-LRSP}.
Thus, understanding the physics that governs the evolution
and sustenance of the activity cycle of the Sun is one of the
long-standing challenges in astrophysics. Thus, it necessitates
imaging its internal layers and accurately characterizing the flows in
the interior.

Modelling the solar dynamo, which is the mechanism that
maintains the Sun's magnetic field, is a real challenge.
Cowling's anti-dynamo theorem \citep{Cowling-1934-MNRAS} states that
a purely axisymmetric flow field cannot by itself sustain an
axisymmetric magnetic field against Ohmic dissipation. The mechanism of
breaking of axisymmetry was provided by \cite{Parker-1955-ApJ} -- who proposed
that convective upflows at different latitudes would experience different
cyclonic twists because of Coriolis force, thus breaking axisymmetry and
providing a mechanism to bypass Cowling's theorem. Subsequent attempts to
model the dynamo led to mean-field electrodynamics. However, dynamo
models are ``yet to recover from the three-way punch''
\citep{Charbonneau-2020-LRSP}, namely, buoyancy effects on magnetic fields,
questions regarding magnetic diffusivity, and the observed
solar differential rotation, which is markedly different from models
that produce solar-like dynamos. Hence, precise estimates of solar internal
flows provide crucial inputs to constrain dynamo models.\\


